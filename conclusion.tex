\section{Conclusions}
\label{conc}
%future work type checking
%mention debugging
In this paper we proposed a Scala library for efficiently implementing the cooperative scheduling behavior of ABS which provides a powerful programming abstraction. 
To provide support for the functional data types required by ABS we embedded the Java run-time system into the Scala programming language. 
We provided a benchmark tailored towards cooperative scheduling that show significant improvements of saving continuations including the call stack as data in memory instead of using a process-oriented approach by means native Java threads. 
We used a second benchmark for a CPU-intensive application to show that this feature does have a negative impact on performance compared to the Akka Actor library.

\par Furthermore having a portable JVM library gives us a basis for industrial adoption of the ABS language and provides ABS as a powerful extension of Scala with support for formal verification, resource analysis and deadlock detection as well as extensions that support real-time programming ~\cite{rabs}, in a software development context. In particular the ABS extension of Scala provide can be integrated with the distributed ABS implementation that exists in the Haskell backend \cite{cloud} to provide a distributed programming model for Actor-based applications.

In future work we plan to extend the library to statically type-check the message submitted via the \textit{send} method in order to prevent the user from running unwanted code on the actors. 
static type checking with respect to actor interfaces

syntactic sugar 

await the proper way

distributed actors

%\par For future work we plan to extend the ABS typechecker to verify the Foreign Language Interface constructs directly in ABS. We also plan to develop a debugger to enable profiling and visualization of concurrent programs written in ABS. 

%We have used the library in modeling real case studies such as the auctioning agent system presented by Dastani and Testerink \cite{bas16}  or applications that already use the formal description capabilities of ABS such as the work of Hahnle and Muschevici in validating railway systems\cite{railway}.


%We have a java library
%We have already benchmarked overhead, but we want to use it for real case studies auctioning and incremental evaluation of railway systems
%Promising candidate fo a basis for industrial adoption of the ABS language.
%Advantage 
%ABS is a rich language with this backend because of the formal analysis, resource analysis and deadlock detection.
%Fromally defined language which provides the use of FA. 
%Extensions for real-time programming and distributed programming.  
%Scala backend for ABS have a full implementation of the functional data types (auctioning system)