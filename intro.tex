\section{Introduction}

%how this scala library is build up
%compiler which translates to scala making use of features for the functional part, source-to-source translation
%underlying java library iwth the thread model
%this compiler supports the FLI

%the actor-based model smotthly integrates with scala

The Actor-based model of computation~\cite{Agha} is particularly tailored to the description of distributed systems. 
Actors represent processes that execute in parallel and interact via asynchronous communication of messages. 
The Abstract Behavioral Specification (ABS)~\cite{abs} Language has been developed for modeling parallel and distributed systems based on the actor model and provides extensive support for formal analysis like functional analysis~\cite{KeY}, resource analysis~\cite{saco} and deadlock analysis~\cite{dead}. Syntactically ABS has a similar structure to Java and Scala languages, offering both functional and object-oriented support.
In this paper, we bring the concepts and advantages of ABS to programming level by implementing a library that can be directly used from Scala and Java. Our solution uses an underlying thread model implemented in Java to support a very powerful feature of ABS, namely cooperative scheduling. Together with this model we provide a source-to-source translation from ABS to Scala to support both the object-oriented and functional programming paradigms of ABS. Most importantly the library has support for a Foreign Language Interface (FLI) that offers ABS as an extension of the Scala language which provides a new concurrency model and corresponding analysis techniques.
% e.g., recent developments have focused on the modeling and analysis of Cloud applications~\cite{Albert}. 


\par In ABS, asynchronous messages are modeled as methods of an actor.
Sending an asynchronous message will schedule the execution of the corresponding method in the callee, returning immediately a future that, upon completion, will hold the result of the method execution, or in case of errors, the thrown exception.
This ``programming to interfaces'' paradigm enables static type checking of message passing at compile time.
This is contrast to the typical approach in Scala where messages are allowed to have any type and thus are only checked at run-time whether the receiver can handle them.

ABS further extends the Actor-based model with a high-level synchronization mechanism that allows actors to suspend in a cooperative manner the execution of the current message and start another queued message. 
The continuation of the suspended message, which is automatically put back in the message queue, may also be assigned an enabling condition. 
A typical use case is awaiting the completion of a future in the same method that has sent the corresponding message.
This simplifies the programming logic by enabling the programmer to process the outcome of an asynchronous method call in the same place where it was sent (in a way similar to dealing with synchronous calls).
Bear in mind that the continuation will be executed in the same actor thread thus (unlike the {\ttfamily onComplete} hooks in Scala that may run in a different thread) pose no threat to actor semantics.
%More details on syntax and behavior of ABS is presented in Section \ref{}. % section 2

%In this paper we introduce and evaluate a new run-time system in Java for the use of ABS as a full-fledged programming language. 
%As such our run-time system allows a component-based design in Java when following the ABS programming methodology.
The main challenge is the development of an efficient and scalable source-to-source implementation of cooperative scheduling. 
%which, in the presence of synchronous calls, gives rise to the suspension of the entire call stack generated from a message.
%The continuation of the suspended message including the stack frame needs to be saved in order for execution to resume properly once the message is rescheduled.
The basic feature of our proposed solution is the implementation of messages by means of lambda expressions (as provided by Java 8), i.e., the method call specified in a message
is translated into a corresponding lambda expression~\cite{lambdas} which is passed and stored as
an object of type Callable or Runnable (depending on whether it returns a result). We show how this basic feature is integrated in a general Scala/Java run-time system for ABS which also caters for synchronous calls that give rise to the suspension of the entire call stack generated from a message.

We describe how we implemented ABS actors and cooperative scheduling using Java thread pools and executor services.  
As elaborated in Section \ref{bench}, we took an approach with negligible performance penalties regarding message passing and cooperative scheduling. 
We describe step by step how to support up to millions of messages,  actors, and suspended continuations.


%This feature forms the basis of a run-time system which manages
%the sending, storage and execution of the messages and their continuations
%which arise because of the cooperative scheduling.
By means of a typical benchmark we evaluate our proposed solution and compare it
to other thread-based implementations in Java or Erlang(a language that uses lightweight threads) of cooperative scheduling.


%related work takka.
% the original version of Akka did not support it, but a recent version provides a first approach to programming to interfaces Takka.
\paragraph{Related Work.} 
In this paper we provide a Scala implementation of the ABS language as a full-fledged Actor-based programming language which features cooperative scheduling and fully supports the "programming to interfaces'' paradigm. 
In contrast, Java libraries for programming actors like Akka~\cite{Akka} mainly provide pure asynchronous message passing which does not support the use of application programming interfaces (API) because, for example, in Akka/Java a message is only typed as a Java Object, so there is no static typing of messages, nor are they part of the actor interface.  A first approach for supporting programming to interfaces was proposed in Takka library~\cite{takka}. 
Distinguishing features of the ABS comprise of high-level constructs for cooperative scheduling which allow the application of formal methods, e.g.,
formal analysis of deadlock~\cite{deadlock}.
The Actor model in Scala~\cite{Scala} does provide a suspension mechanism, but its use is \emph{not} recommended because it actually blocks the whole thread and causes
%its intricate semantics and 
degradation of performance.
It is possible to register a {\em continuation} piece of code to run upon completion of a future, but that may run in a separate thread which however breaks the actor semantics and may cause race conditions inside the actor.

There exist various implementation attempts for cooperative scheduling in ABS.
The approach followed in the Java backend of ABS~\cite{abs,Schafer} and in the Erlang backend~\cite{Erlang} is process-oriented in the sense that sending a message is implemented by the generation of a corresponding process. We will compare these backends with our solution in Java which allows to store messages (as objects of type Callable or Runnable) before executing them (data-oriented).

The focus of our paper is on an efficient implementation of cooperative scheduling in Java.
Implementations in other languages allow for different approaches:
In the Haskell backend for ABS~\cite{Haskell}, for example, the use of continuations allows
to queue a message as a process and dequeue it for execution and the C implementation of cooperative scheduling for the
Encore language~\cite{Encore} uses low-level (e.g., machine code) operations for
storing and retrieving call stacks from memory.

In contrast, in the Actor-based modeling language Rebeca~\cite{Sirjani}, for example,
messages are queued and processed in a run-to-completion mode of execution.
Cooperative scheduling is a powerful means for the expression and analysis
of fine-grained run-time dependencies between messages.

The rest of this paper is organized as follows: In Section~\ref{lang} we give an overview of the main features that the target actor-based model has. Section~\ref{scheme} presents how our solution transitions from a process-oriented approach to a data-oriented approach. Section~\ref{run} describes the implementation the run-time system for the actor-based model. In Section~\ref{bench} we show the experimental evaluation of our solution followed by the conclusions drawn in Section~\ref{conc}.






