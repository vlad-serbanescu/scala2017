

\section{Cooperative Scheduling in ABS}
\label{lang}
%Our reference modeling language for analyzing cooperative scheduling is ABS \cite{abs}, an actor-based modeling language whose semantics offers high-level synchronization mechanisms for parallel and distributed applications. This is a very powerful modeling language with extensive support for concurrent programming \cite{cog}, resource analysis \cite{saco}, deadlock analysis \cite{dead} and remote communication \cite{dis,cloud}. ABS provides high-level constructs for modeling asynchronous communication between actors using messages and cooperative scheduling of these messages by an actor.
%Mahdi:
%{\bfseries the following are not the `focus of this paper' as suggested in the previous sentence}
%These constructs can have optional annotations that define custom schedulers in order to model an actor's specific behavior. Asynchronous method calls can also be annotated with costs and deadlines which can be combined with cooperative scheduling policies to create a very powerful scheduler.
In this section we informally describe the main features of the flow of control
underlying the semantics of cooperative scheduling in the ABS language.
For a detailed description of the syntax of the ABS language we refer to~\cite{abs}.

%
%%await on booleans
%\subsection{Actors and Method Invocations}\label{amc}
%In ABS the concept of an Actor is an object with its own queue and thread of execution.
%Abstracting from the syntax of the actual parameters,
%in ABS an asynchronous invocation of a method \texttt{m} of an actor \texttt{a} is described by a statement
%\texttt{Future f=a!m()}, where f is a future used to store the return value
%(assuming that m contains a return statement).
%In our backend this invocation will be stored in a queue of the called actor a.
%Futures can be passed around as references and provide a get operation
%described by \texttt{f.get} which results in the value returned and blocks
%the current actor if the corresponding method invocation has not yet computed
%the return value.
%
%Again, abstracting from the syntax of the actual parameters,
%ABS features synchronous method calls described in the standard manner
%by statements of the form \texttt{ x=a.m() or x=this.m()} (also assuming here that the method m returns a value). In case of a call to a different actor \texttt{a} the semantics of such a synchronous call can be translated
%by \texttt{Future f=a!m(); x= f.get}, for some future f (of the appropriate type). In case the actor calls its own defined method \texttt{m} the translation depends on whether cooperative scheduling may be encountered and the implementation in our Java backend relies on a specific scheduling policy which preserves the synchronous call stack.
%%In case the called actor \texttt{a }and the caller belong to the same COG such a synchronous call can be translated by \texttt{Future f=a!m( ); x= f.get;resume},
%%where the auxiliary statement \texttt{resume} (as introduced in~\cite{resume})
%
%



%This will be described in more detail in Section~\ref{run}.


%The semantics of ABS allows for synchronous method calls of the form \lstinline{a.m()} or \lstinline|this.m()|. These calls have to block the callee actor until they return. Actors also communicate with each other using asynchronous method invocation. This is written as \lstinline|a!m()| or \lstinline|this.m()| which sends a message to an actor to execute method \lstinline|m()|. The callee actor can then proceed with its execution without blocking. Each ABS actor has a queue that stores all asynchronous invocations coming from other actors as messages and executes them sequentially. An important observation to make here is that in modeling an application, ABS assumes there is no "message overtaking", that is the order of messages delivered from one particular actor to another is the same as the order in which they were generated. The result of a message execution can be captured in an future which the caller actor can use to retrieve the result, but also to block its current execution using \lstinline|f.get| in order to synchronize with the actor completing the future. One may also send the future as parameter of a message, allowing other actors to synchronize and use the results. ABS also has support for grouping Actors intoc and only actors belonging to the same group can communicate synchronously. The semantics of ABS allow this behavior as a call \lstinline|a.m()| can be translated as \lstinline|Future f=a!m(); f.get|. The only issue appears when we have a call like \lstinline|this.m()| as such a translation would result in a deadlock for the actor. In order to have a solution, we will first explain the second language concept of ABS.
%The semantics additionally allows for synchronous method calls that only change the internal state of an actor $\leftarrow$ {\bfseries what does this mean?}. 


%but for the scope of our analysis and implementation simplicity we will assume that each actor is defined in its own group. <-- then I would drop it from here


%The result of an asynchronous method invocation can be captured in an future which the caller actor can use to retrieve the result, but also to block its current execution in order to synchronize with the% callee actor. 
%This is done through the ``future.get" statement that a future supplies. % and has the same semantics as futures in other modeling and programming languages. 


%This is a constraint that our implementation needs to take into account and a synchronization mechanism as will be described in our modeled application later in this section $\leftarrow$ {\bfseries do we? make sure it is not overemphasized here.}. 


%\subsection{Await Construct}
In ABS a statement of the form \texttt{await f?}
suspends the executing process which can only be rescheduled if the method
invocation corresponding to the future \texttt{f} has computed the return value.
Similarly, a statement of the form \texttt{await b}, where \texttt{b} denotes a boolean expression, suspends the executing process which can only be rescheduled if \texttt{b} evaluates to true. Finally a \texttt{suspend} statement invokes cooperative scheduling without a condition. In all cases the executing process can be suspended and another (enabled) message can be scheduled for execution.
In contrast, a statement like \texttt{f.get} does not allow the scheduling of
another process of the same actor. It thus blocks the entire actor. 
Listing~\ref{covssy} gives an example ABS model comparing cooperative scheduling to pure asynchronous communication as in pure actor model.
It is important to observe the call by \lstinline{CooperativeMaster} that is followed by an \texttt{await f?}. This allows the master to send computation requests to the workers and then wait for the results. This call suspends the current message execution of the master, but still allows it to process other messages. In contrast, \lstinline{PureMaster} receives the results in another call (\lstinline{success}) enforcing the programmer to divide the logic in two separate methods.
In Section~\ref{bench2} we will discuss the performance of ABS actors in comparison with the Akka Actor model implementation.


%// Interfaces
%interface IMaster {
%	Unit sendWork(List<Int> list, Int depth, Int priorities);
%	Unit success();
%}
%
%interface IWorker {
%	Unit  nqueensKernelPar(List<Int> list, Int depth, Int priority);
%}
%
%// Classes

\begin{lstlisting}[caption=Cooperative scheduling vs. pure async approach, label=covssy]
class PureMaster (Int numWorkers) implements IMaster {
	Unit sendWorkRoundRobin(...) {
		IWorker worker = chooseWorker();
		worker ! work(...);
	}
	Unit success(Result result) { /* process results */ }
}
class CooperativeMaster (Int numWorkers) implements IMaster {
	Unit sendWorkRoundRobin(...) {
		IWorker worker = chooseWorker();
		Future<Result> fut = worker ! work(...); // start computation
		await fut?;
		// process results
}	}
\end{lstlisting}


%class Worker(IMaster master, Int boardSize) implements IWorker {
%	Unit nqueensKernelPar(List<Int> list, Int depth, Int priority) {
%		Result result;
%		if (...) { 
%			Int i = 0;
%			while (i < boardSize) {
%				if (boardValid(i, ...)) {
%					master ! sendWorkRoundrobin(...); // start a parallel subtask
%				}
%				i = i+1;
%		}	}
%		else {
%			master ! success(result);
%}	}	}

%\begin{lstlisting}[caption=NQueens with cooperative scheduling , label=absex]
%
%class Worker(IMaster master, Int boardSize) implements IWorker {
%	Result nqueensKernelPar(List<Int> list, Int depth, Int priority) {
%		List<Result> results = Nil;
%		if (...) { 
%			List<Future<List<Result>>> futList = Nil;
%  		Int i = 0;
%			while (i < boardSize) {
%				if (boardValid(i, ...)) {
%					Future<List<Result>> fut = master ! sendWorkRoundrobin(...); // start a parallel subtask
%					futList = Cons(fut, futList);
%				}
%				i = i+1;
%			}
%			foreach(fut in futList) {
%				await fut?
%				results = Cons(fut.get, results);
%			}
%		}
%		return results;
%}	}	}
%\end{lstlisting}


%The "await" statement of ABS offers a high level mechanism that controls execution within an actor (or, more generally, a COG) based on its internal state or the availability of the result of an asynchronous call via futures. A statement like \lstinline|await future?| differs from "future.get" 
%
%as it does not block the entire execution of the actor, but instead blocks the current message from which the "await" statement was called and allows the actor to continue with executing other messages that are available in its queue. Similarly, for example \lstinline|await this.boolVar| will suspend the current message executing this statement until the field boolVar is set to true (while another method is being executed). In both uses of the "await" construct the suspended message will be made available once the future is completed or the boolean condition evaluates to true. Now we have a simple solution for translating \lstinline|this.m()| without creating a deadlock by translating it to \lstinline|Future f=a!m(); await f?|, but we still need to preserve the logic and resume from this point once \lstinline|m()| has completed. The basic non-deterministic scheduling policy does not guarantee this unless at the semantic level there is a specific contruct\cite{creol,resume}. 

%Even though a very simple concept, these constructs can simplify the logic of the program significantly, but we will see some of the challenges encountered when implementing this behavior in Sections \ref{comp} and \ref{run}.movesmoves

%- the main concepts: async calls, await on boolean and futures, each object has its own queue.
%- example for coroutines (Paul Klint paper at SEN)

%\subsection{Proof of concept}
%\label{ag}
%Here we sketch the use of ABS in implementing Agent-based modeling languages.
%Agent-based modeling has been shown to be a powerful means to express organizational abstractions of autonomous behavior, including auctioning systems \cite{agent_auction,bas16}.
%However, it is still a major challenge to generate \emph{production} code from high-level agent-based modeling concepts, e.g., the deliberation cycle which integrates goal-oriented computation with an event-based communication approach. 
%To lower the barrier for adoption by industry, Dastani and Testerink \cite{bas16} propose a Java methodology which guides the development of agent-based models. 
%This includes a corresponding library of object-oriented design patterns for modeling agent-based concepts, called OO2APL. 
%
%OO2APL includes a complex middleware for management of the deliberation cycle and the event-based communication between agents. 
%Furthermore, this middleware is tightly coupled with the high-level design patterns for the deliberation cycle and the event-based communication mechanism.
%In contrast, Listing \ref{Agent} shows that modeling agents directly by actors allows to abstract in a concise manner from the underlying implementation of the deliberation cycle and the event-based communication mechanism.
%It provides a high-level design pattern in ABS capturing the code structure for modeling agents with a clear separation of message plans and goal rules. 
%The first three lines introduce the generic data types for goals, beliefs and messages.
%In ABS, one can define the goals, beliefs and messages as abstract data types.
%This makes it easy to define the plans for handling incoming messages and the rules for processing the goals by means of pattern matching against these model-specific data types (lines 14 and 21).
%The general interface of an agent simply consists of a single method for receiving messages. An implementation of this interface consists of a set of beliefs and a set of goals which are initialized upon creation by means of a statement
%\lstinline|Agent a1= new A1(B,G)|.
%This statement assigns to the variable a1 a dynamically generated identity of the newly created agent and initializes its beliefs and goals (by the actual parameters \lstinline|B| and \lstinline|G|).
%Finally, the \lstinline|run()| method (line 27) ensures proactive processing of the goal rules and allows for
%interleaved processing of the messages by means of cooperative scheduling as invoked by the \lstinline|suspend| statement (which is an abbreviation of \lstinline|await true|).
%Thanks to the flexibility of ABS, one can easily adopt variations of the above design pattern.
%For example, placing the \lstinline|suspend| statement inside the \lstinline|for| loop will allow interleaving individual goal rules with message plans.
%Furthermore, application-specific scheduling policies \cite{rabs,cog} may be applied if desired, for example, to give some goals or messages higher priorities.
%The actual instances of this general ABS pattern for the description of the auctioning agents (see Appendix \ref{a1}) additionally use cooperative scheduling for a barrier synchronization between the bidding agents
%and the auctioneer. 
%
%
%
%\begin{lstlisting}[caption= Generic Agent Model, label=Agent]
%data Goal = Goal;
%data Belief = Belief;
%data Message = Message;
%
%interface Agent {
%	Unit message(Message m);
%}
%
%class A1(Set<Belief> init_beliefs, Set<Goal> init_goals)movesimplements Agent {
%	Set<Belief> beliefs = init_beliefs;
%	Set<Goal> goals = init_goals;
%	
%	Unit message(Message m) {
%		case m { 
%			Message =>moves{ }
%			_ => { }
%	}	}
%	
%	Unit goal_rule(Goal g) {
%		case g {
%			Goal => { }
%			_ => { }
%	}	}
%	
%	Unit run() {
%		 while(this.goals != EmptySet) {
%  		    for (g in this.goals) {
%		        this.goal_rule(g);
%		    }
%		  suspend;
%}	}	}
%\end{lstlisting}

%// example of a Main configuration
%{
%    Agent a1 = new A1(set[], set[Goal]);
%}





%\begin{table}
%\begin{tabular}{|l|c|c|}
%	\hline
%	& AuctioneerAgent & BiddingAgent \\
%	\hline
%Goals & Item1, Item2 & Item1 \\
%Beliefs & costs, timing constraints & budget, timing constraints, risk factor \\	
%Received messages (plans) & Bid(item, value) & Announce (item) \\
% & & Sold (item) \\
%Goal Rules & start auction for the goal & \\
%\hline
%\end{tabular}
%\end{table}



%Listing \ref{list:agent} shows a concrete model of an agent  in an auctioning system.
%Here, we define four types of messages sent and received by auctioning and bidding agents.  TODO: more explanation required, i.e., what is the initial goal?
%
%
%\begin{lstlisting}[caption= Agent Model, label=list:agent]
%data Message = 
%Announce(Agent announceCaller, Item  toSell, Price price) | 
%Bid(Agent bidCaller, Item  toBuy, Price bid) |
%Sold(Agent soldCaller, Item soldItem) |
%Result(Set<Agent> winners, List<Price> prices, List<Agent> unhappy);
%
%class BiddingAgent(Set<Belief>  [] , Set<Goal > set[init_goal], Rat risk) implements Agent {
%	Set<Belief>  beliefs = [];
%	Set<Goal> goals = set[init_goal];
%        Rat risk = risk;
%	
%	Unit message(Message m) {
%		case m {
%			Announce(caller, slot, price) =>  { ...   } 
%		        Sold(caller, slot) => { ... }
%		}
%	}
%}
%\end{lstlisting}


